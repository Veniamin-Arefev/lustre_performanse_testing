\section{Обзор существующих решений}
\label{sec:Chapter2} \index{Chapter2}

Многие существующие решения имеют гораздо больший функционал и большое количество зависимостей,
что сильно ограничивает область их применения. Также имеется достаточно большой спект коммерческих
решений с закрытым исходным кодом. Далее будут разобраны несколько утилит с открытым исходным кодом.
После их анализа будет сформирован подход к написанию тестов на каждом узле. А после будет добавлен
параллельный запуск этих программ на всех узлах одновременно.

\subsection{Тесты из официального дистрибутива Lustre}
На официальной сайте распределённой файловой системы Lustre можно найти несколько статей о
тестировании данной файловой системы~\cite{lustre_tests}. В основном там представлены тесты,
направленные на тестирование функциональности файловой системы и различных её компонентов, например: 
\begin{description}
    \item[$\bullet$] \textbf{conf-sanity}.
    
    Набор модульных тестов, которые проверяют инструменты настройки и запускают Lustre с несколькими
    различными настройками для обеспечения правильной работы в необычных конфигурациях системы.
    \item[$\bullet$] \textbf{insanity}
    
    Набор тестов, которые проверяют множественные одновременные условия отказа в обслуживании (выключение) компонентов файловой системы lustre и последующее автоматическое возобновление работоспособности.
    \item[$\bullet$] \textbf{sanity}
    
    Регрессионное тестирование - собирательное название для всех видов тестирования программного обеспечения, направленных на обнаружение ошибок в уже протестированных участках исходного кода. 
    
    Набор регрессионных тестов, которые проверяют работу в штатных условиях эксплуатации, когда всем систему функционируют без сбоев. Они тестируют большое количество необычных операций, которые ранее вызывали проблемы с функциональностью или корректностью данных в Lustre.
    
    \item[$\bullet$] \textbf{sanity-dom}
    
    Проверка корректности данных на серверах метаданных.
    \item[$\bullet$] \textbf{sanity-flr}
    
    Проверка корректности дублирования на уровне файлов.
    \item[$\bullet$] \textbf{sanity-lnet}
    
    Тест на корректность для lnet
    \item[$\bullet$] \textbf{sanity-quota}
    
    Набор тестов, которые проверяют корректронсть работы квот файловой системы. То есть запрет пользователям использовать ресурсов больше, чем им выделено.
    \item[$\bullet$] \textbf{sanity-sec}
    
    Проверяет функции идентификации Lustre, включая сопоставление \textbf{UID/GID}
\end{description}

Эти тесты проверяют множество различных составных частей данной системы, эмулируя
различные ситуации, например, отказ некоторых компонентов или создание множества
параллельных запросов к файловой системе.

Данные тесты стоит запускать, если нагрузочное тестирование показывает очень плохие результаты относительно теоритически рассчитаной пропускной способности.

Также есть и различные тесты производительности: ~ \cite{lustre_tests}
\begin{description}
    \item[$\bullet$] lnet-selftest
    
    Это модуль ядра, который работает через внутренюю сеть файловой системы LNET и сетевые драйверы Lustre (LNDs). Он предназначен для проверки возможности подключения сети Lustre, выполнения
    регрессионных тестов и тестирования производительности сети Lustre.
    \item[$\bullet$] obdfilter-survey
    
    Этот скрипт исследования выполняет последовательный ввод-вывод с различным
    количеством потоков и объектов (файлов).
    \item[$\bullet$] sgpdd-survey
    
    Это исследование может быть использовано для охарактеризования производительности переферийного
    устройства. Он имитирует сервер хранения данных, обслуживающий несколько распределённых между
    логическими томами файлов. Данные, собранные им, могут помочь определить ожидания относительно
    производительности при использовании его в качестве сервера хранения данных Lustre.
    \item[$\bullet$] bonnie++
    
    Bonnie++ тест применяется для создания, чтения и удаления множества небольших файлов, и изменения их метаданных. В резулатате своей работы он даёт пользователю производительность его файловой системы в виде скорости чтения-записи.
    \item[$\bullet$] dbench
    
    Тест Dbench используется для моделирования использования файловой системы N клиентами. Полученные в результате выполнения программы данные могут быть интерпретированы как максимальная пропускная способность операций ввода-вывода данной файловой системы.

    \item[$\bullet$] fsx
    
    Тестировщик файловой системы был разработан вне Lustre и предназначен для
    стресс-тестирования необычных операций ввода-вывода и файловых операций.
    Он проверяет целостность данных после каждого шага. 
    \item[$\bullet$] iozone
    
    Тест Iozone применяется для генерации и измерения различных файловых операций.
    
\end{description}

Данные решения производят хорошее нагрузочное тестирование как отдельных элементов файловой системы,
так и всей системы в целом. В частности все они предоставляют информацию о пиковой и средней скорости чтений-записи файлов на диск.

\subsection{Flexible Input/Output}

В официальном репозитории можно найти много полезной информации.\cite{fio}

Изначально FIO был написан, чтобы избавиться от необходимости писать специальные
программы для тестирования, когда человек хочет протестировать определенную рабочую
нагрузку либо по соображениям производительности, либо для поиска/воспроизведения ошибки.
Процесс написания такого тестового приложения может быть утомительным, особенно если
приходится делать это часто. Следовательно, нужен был инструмент, который мог бы имитировать
заданную загруженность ввода-вывода и не заставлял бы прибегать к самостоятельному написанию
теста снова и снова.

Однако, тестовую рабочую нагрузку определить сложно. Может быть задействовано любое количество
процессов или потоков, и каждый из них может использовать свой собственный способ генерации
ввода-вывода. Может быть несколько потоков, будут выполнять запись в разные места одного файла одновременно или выполнять чтение с использованием асинхронного ввода-вывода. FIO должен
быть достаточно гибким, чтобы имитировать оба этих случая и многое другое.

FIO порождает несколько потоков или процессов, выполняющих определенный тип действий ввода-вывода,
как указано пользователем. FIO принимает ряд глобальных параметров, каждый из которых наследуется
потоком, если не указано иное. Типичное использование FIO заключается в записи используемого в
работе файла, соответствующего нагрузке ввода-вывода, который требуется имитировать.\newline\newline\newline

Все эти решения очень хорошо тестируют производительность для одного узла файловой системы,
но ими нельзя протестировать два или более узлов одновременно, что является неприемлимым для
тестирования производительности файловой системы вычислительного кластера. Именно поэтому на 
их основе нужно сформировать методику тестирования файловой системы для одного узла и впоследствии
её распараллелить. Так же из вывода программы должно быть понятно, на какие элементы системы следует
обратить внимание при поиске проблемного места.

