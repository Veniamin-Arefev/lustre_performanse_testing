\section{Введение}
\label{sec:Chapter0} \index{Chapter0}


В настоящее время суперкомпьютеры, в частности, вычислительные кластеры являются достаточно мощными
инструментами для решения огромного спектра задач ~\cite{Zheng2018InteractiveCO}
~\cite{hoefler2014overview}~\cite{bezrukov2018machine}.
Они применяются в вычислительных целях в том случае,
когда задачу можно разделить на несколько подзадач, каждая из которых может выполняться на отдельном узле.
Большое количество ученых используют их для проведения своих расчётов в научных работах, математическом
моделировании и обработке больших объемов данных.

Данные задачи требуют мощных вычислительных ресурсов,
поэтому суперкомпьютеры стали прекрасным инструментом, который удовлетворяет таким высоким требованиям.
Например:
\begin{description}
    \item[$\bullet$] Различные нефтегазовые компании используют суперкомпьютеры для обработки
    сейсмических данных и нахождений новых месторождений природных ископаемых. Собранные сейсмические морские данные могут составлять терабайты информации и требовать более сотни часов на обработку специальными алгоритмами на машине Factor 2000 (~10 терафлопс в секунду). ~\cite{Meuer2013SupercomputersP}
    \item[$\bullet$] Прогноз погоды занимает около 25 часов при средней производительности в 115
    терафлопс в секунду.~\cite{Open_Forecast_HPC_Model_M04} 
    \item[$\bullet$] Численное моделирование в аэродинамике летательных аппаратов. Вычислительная
    гидродинамика требует производительности хотя бы 50 терафлопс в секунду, чтобы завершаться за
    приемлемое время (порядка нескольких дней).  ~\cite{Meuer2013SupercomputersP}
    \item[$\bullet$] Моделирование населения Галактики во Вселенных Темной энергии: Проект
    Millennium-XXL. Данный проект работал на суперкомпьюере JUROPA с производительностью 275 терафлопс в
    секунду и оперативной памятью 30 терабайт. При этом сгенерировав 100 терабайт данных.  
    ~\cite{Open_Forecast_HPC_Model_M04} 
\end{description}



У каждого суперкомпьютера есть своя файловая система. Файловая система --- это порядок, определяющий
способ организации, хранения и именования данных на носителях информации. Файловая система определяет
формат содержимого и способ физического хранения информации, которую принято группировать в виде файлов.
Обычно используются распределённые файловые системы --- это такие системы, в которых доступ к физически
распределённым данным осуществляется посредством тех же интерфейсов, что и к локальным данным.


Главные преимущества распределенных файловых систем:

Сетевая прозрачность - обеспечение тех же самых возможностей доступа к файлам, распределенным по сети
ЭВМ, которые обеспечиваются в системах разделения времени на централизованных ЭВМ.
~\cite{distributed_operating_systems_course}

Высокая доступность - ошибки систем или осуществление операций копирования и сопровождения не должны
приводить к недоступности файлов.~\cite{distributed_operating_systems_course}


Далее будем рассматривать только распределённую файловую систему lustre, потому что она является
бесплатно распространяемой файловой системой с открытым исходным кодом и часто встречается среди суперкомпьютеров 
из рейтинга top500~\cite{top500}.

Для одновременного запуска программы на всех узлах кластера, синхронизации во время запуска тестов и
последующего сбора результатов тестирования используется Message Passing Interface (MPI) --- программный 
интерфейс (API) для передачи информации, который позволяет обмениваться сообщениями между процессами,
выполняющими одну задачу и расположенными на разных физических машинах. 
MPI является наиболее распространённым стандартом интерфейса обмена данными
в параллельном программировании, существуют его реализации для большого числа компьютерных платформ.
Он используется при разработке программ для кластеров и суперкомпьютеров. Основным средством коммуникации
между процессами в MPI является передача сообщений друг другу.

Когда речь заходит о высокопроизводительных вычислениях сразу вспоминаются самые низкоуровневые
языки программирования, такие как Assembler, С, С++. Для наибольшей универсальности конечной программы
был выбрал язык C, и стандарт языка версии C99.

Поскольку настройкой вычислительных кластеров всегда занимается человек, то последний может совершать
ошибки, вследствие чего в некоторых случаях может наблюдаться резкое падение производительности каких-либо
частей вычислительной системы. В частности, при неправильной конфигурации файловой системы может
происходить уменьшение как средней, так и максимальной скорости чтения-записи файлов. Но эту проблему
можно отследить, если провести нагрузное тестирование файловой системы и выявить при каких условиях
возникают данные проблемы. Но чтобы снегенировать правильную нагрузку, нужно знать какие части системы
могут с ней не справиться, поэтому нужно изучить устройство конкретной файловой системы.

