\section{Постановка задачи}
\label{sec:Chapter1} \index{Chapter1}


Существует вычислительный кластер с возможностью запуска MPI программ. На нём настроена и примонтирована
на каждом вычислительном узле файловая система lustre.

Для того, чтобы понять в каких местах происходит падение производительности и выяснить, происходит ли оно, необходимо провести нагрузочное тестирование, ориентированное на компоненты файловой системы. Если скорость чтения-записи будет падать на каких-либо тестах, то можно обозначить проблемные участки. 

А для того, чтобы провести нагрузочное тестирование, необходимо определить набор тестов, которые
будут использоваться при работе программы. Реализация программы должна удовлетворять следующим требованиям:

\begin{description}
    \item[$\bullet$] Программа должна в автоматическом режиме собирать информацию с узлов кластера,
    анализировать её и выводить конечному пользователю.
    
    \item[$\bullet$] После работы программы не должно оставаться никаких дополнительных файлов, которые
    создавались в процессе тестирования.
    
    \item[$\bullet$] Результаты работы программы не должны разительно отличаться между различными
    запусками (отличие результатов более чем в 2 раза считается достаточным, чтобы можно было говорить об
    ухудшении производительности).

    \item[$\bullet$] Тестирующая программа должна иметь возможность адаптироваться к количеству узлов, на которых будет запущена.
\end{description}
